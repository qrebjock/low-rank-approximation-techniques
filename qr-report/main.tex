
\documentclass{article}

\usepackage[utf8]{inputenc}
\usepackage{natbib}
\usepackage{amsmath}
\usepackage{amsfonts}
\usepackage{mathtools}
\usepackage{amssymb,amsmath, amsthm}
\usepackage{hyperref}
\usepackage{enumitem}
\usepackage{commath}
\usepackage{natbib}

\newcommand{\QED}{~~\rule[-1pt]{6pt}{6pt}}\def\qed{\QED}

% Numbering (by section, then sequential)
\newtheorem{theorem}{Theorem}[section]
\newtheorem{proposition}[theorem]{Proposition}
\newtheorem{definition}[theorem]{Definition}
\newtheorem{lemma}[theorem]{Lemma}
\newtheorem{corollary}[theorem]{Corollary}
\renewenvironment{proof}{\textbf{Proof.}}{\QED\bigskip}
\newtheorem{assumption}[theorem]{Assumption}
\newtheorem*{remark}{Remark}

% text abbrevs
\newcommand{\cf}{{\it cf.}}
\newcommand{\eg}{{\it e.g.}}
\newcommand{\ie}{{\it i.e.}}
\newcommand{\etc}{{\it etc.}}
\newcommand{\ones}{\mathbf 1}
\newcommand{\zeros}{\mathbf 0}

% std math stuff
\newcommand{\reals}{{\mathbb R}}
\newcommand{\sreals}{\scriptsize{\mbox{\bf R}}}
\newcommand{\integers}{{\bf Z}}
\newcommand{\naturals}{{\bf N}}
\newcommand{\eqbydef}{\mathrel{\stackrel{\Delta}{=}}}
\newcommand{\complex}{{\mbox{\bf C}}}
\newcommand{\symm}{{\mbox{\bf S}}}  % symmetric matrices
\DeclareMathOperator{\Tr}{Tr}

% probability stuff
\newcommand{\Expect}{{\bf E}}
\newcommand{\Normal}{\mathcal{N}}
% \newcommand{\Expect}{\textstyle\mathop{\bf E}}
\newcommand{\Median}{\textstyle\mathop{\bf M}}
\newcommand{\Prob}{\mathop{\bf Prob}}
\newcommand{\erf}{\mathop{\bf erf}}
\newcommand{\indep}{\perp \!\!\! \perp}
\newcommand\independent{\protect\mathpalette{\protect\independenT}{\perp}}
\def\independenT#1#2{\mathrel{\rlap{$#1#2$}\mkern2mu{#1#2}}}

% \DeclarePairedDelimiter\abs{\lvert}{\rvert}
% \DeclarePairedDelimiter\norm{\lVert}{\rVert}

% paper specific stuff
% \newcommand{\diag}{\mathop{\bf diag}}
\DeclareMathOperator{\diag}{diag}
\DeclareMathOperator{\grad}{grad}
\DeclareMathOperator{\hess}{Hess}
\DeclareMathOperator{\dist}{dist}
\DeclareMathOperator{\acosh}{acosh}
\DeclareMathOperator{\PT}{PT}
\DeclareMathOperator{\rank}{rank}
\DeclareMathOperator{\sym}{Sym}
\DeclareMathOperator{\sksym}{Skew}
\DeclareMathOperator{\nullspace}{null}
\DeclareMathOperator{\vecspan}{span}
\DeclareMathOperator{\proj}{proj}
% \newcommand{\norm}[1]{\left\lVert#1\right\rVert}
\DeclarePairedDelimiterX{\inner}[2]{\langle}{\rangle}{#1, #2}

%%% Local Variables:
%%% mode: latex
%%% TeX-master: "main"
%%% End:

% set page geometry
\usepackage[verbose=true,letterpaper]{geometry}
\AtBeginDocument{
  \newgeometry{
    textheight=9in,
    textwidth=5.5in,
    top=1in,
    headheight=12pt,
    headsep=25pt,
    footskip=30pt
  }
}

\title{Leverage score sampling}
\author{Quentin Rebjock}
\date{\today}

\begin{document}

\maketitle

\section{Notations}\label{sec:notations}

We note $I_n$ the identity matrix of size $n$, or simply $I$ when the size is
inferred by context.
We note $e_i$ the $i$th canonical vector, that is the $i$th column of the
identity whose size is inferred by context.
For any matrix $A \in \reals^{m \times n}$ we note $a_1, \dots, a_n$ its columns
and $A_{i,:}$ its $i$th row.

\section{Leverage scores analysis}\label{sec:questions}

\subsection{Problem statement}\label{subsec:probstat}

Let $A \in \reals^{m \times n}$ be a matrix of rank $r$ whose columns are noted
$a_1, \dots, a_n$.
We study the \textit{leverage scores} of $A$, that is the quantity
\begin{align*}
  \ell_i(A) = a_i^\top (A A^\top)^\dagger a_i
\end{align*}
for the $i$th column of $A$.
We also consider the \textit{Ridge} leverage scores, which are regularized by a
parameter $\lambda > 0$ as follows
\begin{align*}
  \ell_{i,\lambda}(A) = a_i^\top (A A^\top + \lambda^2 I)^{-1} a_i.
\end{align*}

\subsection{Preliminaries}\label{subsec:preliminaries}

Before analyzing the scores we derive convenient expressions for the columns of
$A$ and the leverage scores in terms of its singular value decomposition (SVD).
We note $A = U \Sigma V^T$ the thin SVD of $A$,
where $U \in \reals^{m \times r}$ and $V \in \reals^{n \times r}$ have
orthogonal columns, and
\begin{align*}
  \Sigma =
  \begin{bmatrix}
    \sigma_1 & & \\
    & \ddots & \\
    & & \sigma_r
  \end{bmatrix}
\end{align*}
is a diagonal matrix such that $\sigma_1 \geq \cdots \geq \sigma_r > 0$.
Then the symmetric matrix $A A^\top$ depends only on the left singular vectors
$U$, and
\begin{align*}
  \begin{cases}
    AA^\top = U \Sigma^2 U^\top\\
    (AA^\top)^\dagger = U \Sigma^{-2} U^\top,
  \end{cases}
\end{align*}
where $\dagger$ denotes the Moore–Penrose pseudo-inverse.
Moreover it holds that for any index $i$, the $i$th column of $A$ can be written
as
\begin{align}\label{eq:acol}
  a_i &= U \Sigma V^\top e_i\nonumber\\
      &= U \Sigma V_{i,:},
\end{align}
from which we deduce the equality
\begin{align}\label{eq:ellid}
  \ell_i(A) &= a_i^\top (A A^\top)^\dagger a_i\nonumber\\
            &= V_{i,:}^\top \Sigma^\top \Sigma^{-2} \Sigma V_{i,:}\nonumber\\
            &= V_{i,:}^\top I_r V_{i,:}\nonumber\\
            &= \sum_{j = 1}^r v_{ij}^2.
\end{align}

\subsection{Analysis}\label{subsec:analysis}

\subsubsection{Question 1}\label{subsubsec:q1}

\begin{enumerate}[label=\alph*)]
\item As $V$ has orthogonal columns there exist a matrix $V_\bot$ with
  orthonormal columns such that $\overline{V} =
  \begin{bmatrix}
    V & V_\bot
  \end{bmatrix} \in \reals^{n \times n}
  $ is orthogonal.
  From this and identity~\eqref{eq:ellid} we deduce that
  \begin{align*}
    \ell_i(A) &= \sum_{j = 1}^r v_{ij}^2\\
              &\leq \sum_{j = 1}^n \overline{v}_{ij}^2\\
              &\leq 1.
  \end{align*}
\item We show that the sum of all leverage scores equals the rank of $A$.
  From identity~\ref{eq:ellid}, we compute
  \begin{align*}
    \sum_{i = 1}^n \ell_i(A) &= \sum_{i = 1}^n \sum_{j = 1}^r v_{ij}^2\\
                            &= \sum_{j = 1}^r \sum_{i = 1}^n v_{ij}^2\\
                            &= r,
  \end{align*}
  where the last equality comes from the fact that $V$ has orthonormal columns
  (with unit norm).
\item Let $j$ be such that the column $a_j$ is orthogonal to all others.
  We want to prove that $\ell_j(A) = 1$.
  Without loss of generality, we may
  assume that $j = 1$ by permuting the columns of $A$.
  Identity~\eqref{eq:acol} gives that for any index $i > 1$,
  \begin{align*}
    a_1^\top a_i &= V_{1,:}^\top \Sigma^2 V_{i,:}\\
    &= 0,
  \end{align*}
  which yields
  \begin{align}\label{eq:orthocol}
    V \Sigma^2 V^\top =
    \begin{bmatrix*}
      c & \zeros^\top\\
      \zeros & *
    \end{bmatrix*} \in \reals^{n \times n},
  \end{align}
  where $c$ is some positive number.
  Let $V_\bot \in \reals^{n \times (n - r)}$ be such that $\overline{V} =
  \begin{bmatrix}
    V & V_\bot
  \end{bmatrix} \in \reals^{n \times n}
  $ is orthogonal.
  Then multiplying Equation~\eqref{eq:orthocol} by $V_\bot$ gives that
  \begin{align*}
    \begin{bmatrix*}
      c & \zeros^\top\\
      \zeros & *
    \end{bmatrix*} V_\bot = \zeros_{n \times (n - r)},
  \end{align*}
  which implies that the first line of $V_\bot$ is null (as $c$ is positive).
  Since $\overline{V}$ is orthogonal, we deduce that $\norm{V_{1,:}}^2 = 1$,
  which yields $\ell_1(A) = 1$ according to identity~\eqref{eq:ellid}.
\item Proven in Section~\ref{subsec:preliminaries}, Equation~\eqref{eq:ellid}.
\end{enumerate}

\subsubsection{Question 2}\label{subsubsec:q2}

Let $\lambda \geq 0$  be a regularization parameter for the Ridge
leverage scores.
Using the fact that $U$ has orthonormal columns, the matrix $A A^\top +
\lambda^2 I = U \Sigma^2 U^\top + \lambda^2 I$ can be factorized as $U\left(
  \Sigma^2 + \lambda^2 I \right) U^\top$.
From this we deduce that
\begin{align*}
  \ell_{i,\lambda}(A) &= V_{i,:}^\top \Sigma \left( \Sigma^2 + \lambda^2 I \right)^{-1} \Sigma V_{i,:}\\
                      &= V_{i,:}^\top \Delta V_{i,:}\\
                      &= \sum_{j = 1}^r \frac{\sigma_j^2}{\sigma_j^2 + \lambda^2} v_{ij}^2
\end{align*}
where
\begin{align*}
  \Delta = \Sigma \left( \Sigma^2 + \lambda^2 I \right)^{-1} \Sigma =
  \begin{bmatrix}
    \frac{\sigma_1^2}{\sigma_1^2 + \lambda^2} & & \\
    & \ddots & \\
    & & \frac{\sigma_r^2}{\sigma_r^2 + \lambda^2}
  \end{bmatrix}
\end{align*}
is an $r \times r$ diagonal matrix.

Now let $A \in \reals^{m \times n}$ (with $m \geq n$) be a matrix such that
\begin{align*}
  \sigma_1 \geq \cdots \geq \sigma_k \gg \lambda \gg \sigma_{k + 1} \geq \cdots \geq \sigma_n > 0.
\end{align*}
Then $A$ is full-rank ($r = n$) since $\sigma_n > 0$.
We conclude using Question 1.d (Section~\ref{subsubsec:q1}) that all the
leverage scores are equal to $1$, as the rows of $V$ all have unit norm.
However,
\begin{align*}
  \delta_i \approx
  \begin{cases}
    1 \text{ when } 1 \leq i \leq k,\\
    0 \text{ when } k < i \leq n,
  \end{cases}
\end{align*}
which yields that
\begin{align*}
  \ell_{i,\lambda}(A) &= \sum_{j = 1}^n \delta_j v_{ij}^2\\
                      &\approx \sum_{j = 1}^k v_{ij}^2
\end{align*}
which is the squared norm of the $i$th row of the right singular vectors of
the truncated SVD of $A$.

\section{Experiments}\label{sec:experiments}

We wish to approximate the range of $A$ by a matrix with $k$ columns, where we
think of $k$ as a small number.
More specifically we sample from the columns of $A$ independently at random,
with replacement.
A positive weight is assigned to every column and its probability to be sampled
is proportional to this weight.
We implement the following sampling strategies in Julia:
\begin{itemize}
\item Weights equal to Ridge leverage scores, with $\lambda = 10^{-4}$;
\item Uniform sampling;
\item Weights equal to the squares of the norms of the columns of $A$.
\end{itemize}
We consider the $100 \times 100$ Hilbert matrix $A$, which has rank $\leq 20$
(numerical instabilities prevent from computing the exact rank).
For every $k \in [50]$, we sample columns to form a matrix $C$ according to the
three strategies described above, and measure the error $\|A - C C^\dagger A\|$.
We repeat this process $200$ times and plot the average error in
Figure~\ref{fig:stratcomp}.
\begin{figure}[h!]
  \centering
  \begin{minipage}[b]{\textwidth}
    \includegraphics[width=\textwidth]{../code/plot.pdf}
  \end{minipage}
  \caption{
    Average error $\|A - C C^\dagger A\|$ over $200$ trials for each method as a
    function of $k$, where $C$ contains $k$ columns sampled from $A$
    independently and with replacement.
    The black curve is the best error that can be obtained from a matrix with
    $k$ columns.
  }\label{fig:stratcomp}
\end{figure}
We see that the Ridge leverage scores sampling yields the lowest error in
average.
However sampling columns proportionally to the squared norms performs reasonably
well at a much lower computational cost.

\bibliographystyle{plainnat}
\bibliography{references}

\end{document}